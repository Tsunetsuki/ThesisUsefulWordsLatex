\section{Basic concepts of Natural Language Processing}
 [basic concepts, and why they matter to this work.
  I will add anything that that the target audience is either not familiar with, or where the function in my method is not obvious]

\begin{description}
	\item [AI model] Container of functional knowledge, in this case linguistic.
	\item [General NLP tasks] a) Way for AI model to engage with language, b) Test of language ability.
	\item [Explainable AI] Extraction of knowledge from AI model.
	\item [Transformer Attention mechanism] Way of finding patterns in functional knowledge of some AI models.
	\item [Tokenizer] To split continuous text into distinct words, which we can make statistics from, mask in text etc.
\end{description}

This paper aims to exploit these significant developments to gain insights into which words can provide second language learners with the most utility.

\section{Definitions of terms}
 [Definition of "utility" of vocabulary]

\section{State of the Art (of vocabulary selection)}
\subsection{Non-computational methods}
How is vocabulary order selected by current language teaching tools:
Not context-specific in most cases
offers only one order,
investigate if there are any automatic approaches besides frequency]

[can present methods for extracting useful words from text]
can be similar in either method or goal.
goal is finding useful words for language learning
method is extracting important words from text via XAI

\cite{heChoosingWordsTeach2019} cites concepts of frequency, usefulness and difficulty of words as widely accepted criteria to decide when to teach it.
However, the literature is not clear on how usefulness is defined outside of "human intuition" \cite{heChoosingWordsTeach2019}.
Furthermore, usefulness is presented as independent from frequency, making it more unclear still how it may be defined.
The advantage of human intuition is that humans can understand the nuances of texts better than computers, especially before the invention of large AI models tackling NLP tasks.
Relying on human intuition to evaluate word utility has two main drawbacks compared to computational methods:
\begin{itemize}
	\item Difficult to put a concrete number on word.
	\item Evaluating many languages would necessitate human experts for each language, necessitating expensive studies.
\end{itemize}

\subsection{Computational methods}
These methods generate vocabulary lists by using corpora and computer-aided language processing to compile vocabulary lists.
These are closer to the method I will

\begin{description}
	\item [Raw frequency of words in corpus]
	      A simple ordering of words by how often they appear in a corpus.
	\item [Frequency with stopwords filtered out]
	      The same as frequency, but filtering out known stopwords from the resulting lists.
	\item [TF-IDF]
	      How often the words appear in a target document but divided by their frequency in a more generic corpus.
	      This metric is typically used to employ the most relevant words in documents for identifying keywords that express best its core topic.
\end{description}

\subsection{Issues with current methods}
Current methods do not exploit recent developments in AI technology and thus suffer from several shortcomings:
In general, all of them essentially only count words without taking into account their relationship between each other:

Frequency: The most frequent words in texts are often words that carry little meaning by themselves, such as "a", "the", "of" in English.
While these may appear in many texts, they are not useful in determining their meaning.
TF-IDF: This metric has been used successfully to find the words that give the best hints at a text's topic.
However, it does not take into account and semantic relationships between the words in a text.
Thus, learning words by aggregating TF-IDFs on multiple texts may aid in identifying the topic of texts, but not at finding out what the message conveyed about the topic is.
"not" is a highly frequent word in English and thus will have a low TF-IDF score in most documents. But it is essential to know, as it can completely invert the meaning of a sentence.

