\section{Motivation}
Learning a second language involves many different skills, often categorized into listening, reading, speaking, and writing.
Another categorization may be vocabulary, grammatical skills, the ability to understand known words in various accents, understanding language when spoken at a fast speed.
One skill that is required for any of these if the knowledge of vocabulary in the target language.
A person with basic grammatical skills but no vocabulary has no ability to express themselves or understand anything which they hear around them.
On the other hand, a person familiar with rudimentary vocabulary but no grammatical knowledge may struggle with understanding complex sentences and sound unnatural when speaking, but can at least make sense of short phrases and express themselves.
Thus, basic knowledge of vocabulary is clearly one of the most essential skills for using a language.
This raises the question of which vocabulary should be learned first when starting out on the journey of language acquisition.
Leaving aside vocabulary that might not be desirable at all, this question may be reduced to:

\textbf{What order gives the learner the best set of words to understand and communicate in the language as quickly as possible?}


...

\section{Background}
\subsection{Requirements for calculating word utility from existing data}

\begin{description}
	\item [Large word corpora in many languages]
	\item [Hardware capabilities]
    \item [Software is able to process text on semantic level]
	      Text processing software and tools have existed since the 1980s, however before the advent of neural networks and other artificial intelligence methods, their capabilities were mostly bereft of any semantic understanding of the text processed: Using exclusively manually crated tools, it is possible to create a program which recognizes that \textit{table} and \textit{tables} derive from the same word, but much harder to recognize that there is any connection whatsoever between \textit{table} and \textit{chair}. The language processing capabilities of any human dwarf those of these early tools.
\end{description}

\subsection{Developments in Natural Language Processing}
\begin{description}
	\item [Sophisticated tokenizers]
	\item [Wordnets]
	\item [Neural networks trained in NLP tasks]
	\item [Explainable AI]
\end{description}

This paper aims to exploit these significant developments to gain insights into which words can provide second language learners with the most utility. 

