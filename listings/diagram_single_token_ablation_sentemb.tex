\begin{center}
	\scalebox{0.75}{
		\begin{tikzpicture}[node distance=1.8cm and 2.5cm, every node/.style={font=\small}]

			% Styles
			\tikzstyle{process} = [rectangle, rounded corners, minimum width=3.5cm, minimum height=1cm, text centered, draw=black, fill=blue!10, text width=8.5cm]
			\tikzstyle{arrow} = [thick, ->, >=Stealth]
			\tikzstyle{data} = [align=left, text width=8.5cm, anchor=west]

			% Process Nodes
			\node (n1) [process] {Input sentence};
			\node (n2) [process, below=of n1] {Tokenize};
			\node (n3) [process, below=of n2] {Merge tokens, filter named entities and special tokens};
			\node (n4) [process, below=2cm of n3] {Make sentence variations by ablating merged tokens};
			\node (n5) [process, below=3cm of n4] {Run model on baseline and variations };
			\node (n6) [process, below=of n5] {Calculate cosine distance between variation vectors and baseline};

			% Anchor point to align all data nodes horizontally
			\node (anchor) [right=of n1] {};

			% Data Nodes, all vertically stacked but aligned to 'anchor'
			\node (d1) [data] at (anchor) {\texttt{"Abraham Lincoln faced enmity in 1863."}};
			\node (d2) [data, right=of n2] {
				\texttt{['[CLS]', 'abraham', 'lincoln', 'faced', 'en', '\#\#mity', 'in', '1863', '.', '[SEP]']}
			};
			\node (d3) [data, right=of n3] { \texttt{['faced', 'enmity', 'in']} };
			\node (d4) [data, right=of n4] {
				\textbf{Baseline:} \\
				\quad \texttt{"abraham lincoln faced enmity in 1863."}\\
				\textbf{faced:} \\
				\quad \texttt{"abraham lincoln enmity in 1863."}\\
				\textbf{enmity:} \\
				\quad 	\texttt{"abraham lincoln faced in 1863."}\\
				\textbf{in:} \\
				\quad \texttt{"abraham lincoln faced enmity 1863."}\\
			};
			\node (d5) [data, right=of n5] {
				\textbf{Baseline:} $\vec{r}$ \\
				\textbf{faced:} $\vec{a}$\\
				\textbf{enmity:} $\vec{b}$\\
				\textbf{in:} $\vec{c}$\\
			};
			\node (d6) [data, right=of n6] {
				\textbf{faced:} \texttt{0.5}\\
				\textbf{enmity:} \texttt{0.7}\\
				\textbf{in:} \texttt{0.3}\\
			};

			% Arrows between process nodes (left column)
			\draw [arrow] (n1) -- (n2);
			\draw [arrow] (n2) -- (n3);
			\draw [arrow] (n3) -- (n4);
			\draw [arrow] (n4) -- (n5);
			\draw [arrow] (n5) -- (n6);

		\end{tikzpicture}
	}
\end{center}

