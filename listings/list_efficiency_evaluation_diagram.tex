\begin{center}
	\scalebox{0.75}{
		\begin{tikzpicture}[node distance=1.8cm and 2.5cm, every node/.style={font=\small}]

			% Styles
			\tikzstyle{process} = [rectangle, rounded corners, minimum width=3.5cm, minimum height=1cm, text centered, draw=black, fill=blue!10, text width=8.5cm]
			\tikzstyle{arrow} = [thick, ->, >=Stealth]
			\tikzstyle{data} = [align=left, text width=8.5cm, anchor=west]

			% Process Nodes
			\node (n1) [process] {Input sentence};
			\node (n2) [process, below=of n1] {Tokenize};
			\node (n3) [process, below=of n2] {Merge tokens, filter named entities and special tokens};
			\node (n4) [process, below=of n3] {Make sentence variations by ablating merged tokens};
			\node (n5) [process, below=of n4] {Run model for variations, evaluate performances};
			\node (n6) [process, below=of n5] {Calculate difference to baseline in performance};

			% Anchor point to align all data nodes horizontally
			\node (anchor) [right=of n1] {};

			% Data Nodes, all vertically stacked but aligned to 'anchor'
			\node (d1) [data] at (anchor) {\textbf{Baseline}: "Abraham Lincoln faced enmity in 1863."\\
				\textbf{Variation}: "Abraham Lincoln enmity 1863."
			};
			\node (d2) [data, right=of n2] {
			['[CLS]', 'abraham', 'lincoln', 'faced', 'en', '\#\#mity', 'in', '1863', '.', '[SEP]']
			};
			\node (d3) [data, right=of n3] { ['faced', 'enmity', 'in'] };
			\node (d4) [data, right=of n4] {
				\textbf{Baseline:} "abraham lincoln faced enmity in 1863."\\
				\textbf{faced:} "abraham lincoln enmity in 1863."\\
				\textbf{enmity:} "abraham lincoln faced in 1863."\\
				\textbf{in:} "abraham lincoln faced enmity 1863."\\
			};
			\node (d5) [data, right=of n5] {
				\textbf{Baseline:} 0.8\\
				\textbf{faced:} 0.5\\
				\textbf{enmity:} 0.3\\
				\textbf{in:} 0.7\\
			};
			\node (d6) [data, right=of n6] {
				[('faced', 0.3), ('enmity', 0.5), ('in', 0.1)]
			};

			% Arrows between process nodes (left column)
			\draw [arrow] (n1) -- (n2);
			\draw [arrow] (n2) -- (n3);
			\draw [arrow] (n3) -- (n4);
			\draw [arrow] (n4) -- (n5);
			\draw [arrow] (n5) -- (n6);

		\end{tikzpicture}
	}
\end{center}

